\begin{abstract}
The problem of competition, cooperation and the emergence of collective behaviour in the presence of limited resources is quite general, and one of cornerstones of evolutionary dynamics both in the natural and artificial worlds alike. For instance, trading is nowadays mainly performed by algorithms which act autonomously and form an ecosystem of their own. 
These models exhibit complex phenomena, governed by various parameters that describe the quantity and the quality of agents involved. We study how these parameters influence the efficiency of the models, measured as efficient distribution of limited resources, and how the additional information like vicinity, it's structure, the number and the cognitive abilities of participants modify these models efficiency. 
\par
We extend the classical prototype of such environment, i.e,  minority games, by allowing agents to process  additional information. We analyse how this exchange of information affects the dynamics of the system. The main focus is put on the role of the information from each agents vicinity and to study the influence of different community structures on the model. We investigate several topologies like simple patch vicinity, von Neumann vicinity, small world, scale-free networks and a hierarchical small world network. 
\par
We have found that there is a distinct relation between the structure and dimension of the vicinity, ie. the information given to each agent, and the efficiency of the model. These results could be used to optimise any kind of distributed algorithmic ecosystem that has a finite resources and needs an efficient use of it.
\par
We further investigate how these finding can be used in algorithmic ecosystems. The context within which we elaborate some of our ideas is high-frequency financial markets, that are run by an enormous number of trading algorithms. Among other possible applications we consider optimizing the routing protocols for Delay Tolerant Networks, more efficient smart-grid energy systems and so on.
\end{abstract}

