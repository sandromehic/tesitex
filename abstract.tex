\begin{abstract}
Problems of limited resources in competitive systems are encountered and elaborated by humans and algorithms alike every day. These models present complex behaviour, governed by various parameters that describe the quantity and the quality of agents involved. In this thesis we study how these parameters influence the efficiency of the models, measured as efficient distribution of limited resources, and how the additional information like vicinity, it's structure, the number and the cognitive abilities of participants modify models efficiency. 
\par
We use an agent-based approach to model these competitive systems, while certain variations of minority rule are used to implement the frustration inside the model, Starting with classical minority games, based on a number of agents with deterministic strategies, we expand the model by implementing additional information and analyse how it impacts the evolution of the system. Mainly we focus on the information found within each agents vicinity and study the influence of different community structures on the model. Main vicinity structures used are: simple patch vicinity, von Neumann vicinity, small world and scale-free networks. 
\par
We further investigate how these finding can be used in algorithmic ecosystems that are characterized by competitive nature and finite resources. The context within which we elaborate some of our ideas is the high-frequency financial markets, that are run by an enormous number of trading algorithms. Among other possible applications we consider optimizing the routing protocols for Delay Tolerant Networks, more efficient smart-grid energy systems and so on.
\par
We have found that there is a distinct relation between the structure and dimension of the vicinity, ie. the information given to each agent, and the efficiency of the model. These results could be used to optimise any kind of distributed algorithmic ecosystem that has a finite resources and need an efficient use of it.
\end{abstract}