\chapter{Introduction}

\cite{challet2013minority}

\section{Struttura della tesi}
Quello che ci si è proposti di fare in questo lavoro di tesi è la costruzione di un'architettura che combinasse input visivi ed output derivanti da processi di visione primitiva al fine di ottenere risultati percettivi privi di ambiguità, prendendo spunto dal funzionamento del sistema visivo umano.
Si è cercato cioè di determinare i fattori più importanti che influiscono nella generazione delle immagini e delle relazioni che intercorrono fra di loro. Una volta acquisite, queste relazioni esse possono essere usate nei processi inversi per ottenere le informazioni che hanno contribuito a creare le immagini.
In particolare ci siamo occupati del problema dello \textit{shape from shading}, che ha l'obiettivo di ricostruire la struttura 3D di un oggetto a partire dalle sole informazioni di ombreggiatura.
Descriveremo questa architettura nel capitolo 2, dopodiché procederemo ad illustrare qual'è lo stato dell'arte dello shape from shading nel capitolo 3. 
Nel capitolo 4 riporteremo il lavoro svolto e i risultati prodotti, concludendo nel capitolo 5 con proposte e sviluppi futuri.